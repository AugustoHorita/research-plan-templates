%---------------------------------------------------------------------------- 
%
%  $Description: General Research Plan Structure $
%
%  $Author: dloubach, with great contribution from rbonna $
%  $Release Date: December 20, 2016 $
%
%  O Projeto de Pesquisa deve demonstrar claramente os desafios científicos ou 
%  técnicos a serem superados pela pesquisa proposta, os meios e métodos para 
%  isso e a relevância dos resultados esperados para o avanço do conhecimento 
%  na área.
%  Máximo 20 páginas.
%
%  The research project shall clearly demonstrate the scientific or techinical
%  chalenges to be overcame by the proposed research, as well as the ways and
%  methods to achieve so and moreover the expected results relevance to advance
%  the area knowledgement.
%  20 pages maximum.
%---------------------------------------------------------------------------- 
\documentclass[12pt,a4paper]{article}

% [portuguese|english], label=[pt_br, en_usa], level=[undergrad | msc | phd]
\usepackage[english, label=en_usa, level=phd]{general-settings}

% [on | off]
\usepackage[on]{review-settings}

%---------------------------------------------------------------------------- 

% title definitions
\newcommand{\researchTitle}{The Great Title of your Great Research}
\newcommand{\researchTitleOtherLanguage}{O Excelente Título para a sua Excelente Pesquisa}

% research level definitions
\newcommand{\studentName}{Your name goes here}
\newcommand{\advisorName}{Denis S. Loubach}

% figures path
\graphicspath{{./figs/}}

% space between lines
\linespread{1.3}

\begin{document}
% our title page
\makeourtitle

% contents
\tableofcontents

\newpage
\section*{Resumo}
\label{sec:resumo}
\Dloubach{As you're writting in Brazil, we need to write something in Portuguese (at least this part)}
Aqui vai o resumo do seu trabalho...\\%


\textbf{Palavras chave --} palavra-chave$_1$; ...; palavra-chave$_n$.


\newpage
\section*{\textit{Abstract}}
\label{sec:abstract}
\textit{
  Your abstract goes here.
}\\%

\textit{\textbf{Keywords --} keyword$_1$, ..., keyword$_n$.}


\newpage
\section{\sectionI}
\label{sec:introducao}
Aqui vai a introdução do seu trabalho, contextualizando o leitor no assunto principal da sua pesquisa.

Exemplo de figura \ref{fig:perf_flex}:

\begin{figure}[ht]
	\centering
	\tikzset{my node/.style={circle, inner color=#1!20, outer color=#1!50,
	  draw=#1!75, text=black}}
	\begin{tikzpicture}
	  \node[my node = red] at (1,5) {GPPs};
	  \node[my node = blue] at (9,1) {ASICs};
	  \node[my node = purple] (SA) at (5.5,3.5) {FPGAs};
		\draw[thick,-latex] (0,0) -- (10,0)
			node[pos=1,below left]{performance};
	  \draw[thick,-latex] (0,0) -- (0,6)
			node[pos=1,above left,rotate = 90]{flexibilidade};
		\draw[color=gray,double,thick,-latex'] (SA)++(40:1) -- ++(40:1);
	\end{tikzpicture}
	\caption{Performance vs. flexibilidade em GPPs, ASICs e FPGAs.
	  Adaptado de \cite{Bobda2007a}.}
	\label{fig:perf_flex}
\end{figure}


\section{\sectionII}
\label{sec:trabalhos-relacionados}
Esta seção deve apresentar, de forma breve, os principais trabalhos relacionados aos conceitos fundamentais da sua pesquisa.

Exemplo:\\%
O trabalho \cite{Loubach2016a} introduz um \textit{design} de reconfiguração em tempo de execução para sistemas embarcados reconfiguráveis visando sistemas aviônicos. Leva-se em consideração principalmente a performance e o consumo de energia para reconfiguração parcial e total. Baseado nos resultados obtidos para consumo de energia, tempo de reconfiguração e tamanho dos \textit{bitstreams} (bits de configuração de FPGAs), pode-se considerar que sistemas embarcados reconfiguráveis em tempo real são viáveis para serem utilizados em sistemas aviônicos de gerações futuras.

Busque dar uma visão ampla de trabalhos básicos fundamentais e também trabalhos mais recentes.


\section{\sectionIII}
\label{sec:objetivo}
O objetivo desse trabalho de pesquisa consiste em ``o que vc vai fazer de pesquisa'', visando ``qual o benefício principal esperado'' .


\section{\sectionIV}
\label{sec:escopo-da-pesquisa}
O escopo deste trabalho de pesquisa concentra-se nos seguintes tópicos principais:

\begin{itemize}
\item Estudar ``isso'';
\item Estudar ``aquilo'';
\item Definição de ``modelo/projeto/desing/.../o que vc estiver fazendo de principal''; e
\item Testes, verificação e validação da ``proposta'' utilizando um estudo de caso.
\end{itemize}


\section{\sectionV}
\label{sec:plano-trabalho-cronograma}
Deve se apresentar aqui as principais atividades desse plano de trabalho, bem como o cronograma proposto para realização destas atividades.

\begin{enumerate}[A.]
  \item Disciplinas obrigatórias;
  \item Revisão bibliográfica;  
  \item Estudo da ``técnica x, teorema y, modelos z'';  
  \item Estudo e testes básicos das plataformas de hardware;  
  \item Elaboração de uma proposta de ``core da sua pesquia'';
  \item Elaboração de experimentos práticos de hardware;  
  \item Testes e verificação do modelo proposto; e    
  \item Validação e refinamentos do modelo proposto.   
\end{enumerate}

\begin{table}[ht]
  \centering
  \caption{Exemplo de cronograma de atividades dividido em trimestres.
    \label{tab:cronograma}}
  \begin{tabular}{|c||c|c|c|c||c|c|c|c||c|c|c|c||c|c|c|c|}
    \hline
    & \multicolumn{4}{|c||}{2016 - 2017}
    & \multicolumn{4}{|c||}{2017 - 2018}
    & \multicolumn{4}{|c||}{2018 - 2019}
    & \multicolumn{4}{|c|}{2019 - 2020} \\ \hline \hline
    Trimestre & 3 & 4 & 1 & 2 & 3 & 4 & 1 & 2 & 3 & 4 & 1 & 2 &
      3 & 4 & 1 & 2 \\ \hline \hline
    A & x & x & \ck & \ck & & & & & & & & & & & & \\ \hline
    B & x & x & \ck &  & & & & & & & & & & & & \\ \hline
    C & & & \ck & \ck & & & & & & & & & & & & \\ \hline
    D & & & \ck & \ck & \ck & & & & & & & & & & & \\ \hline
    E & & & & & \ck & \ck & \ck & \ck & & & & & & & & \\ \hline
    F & & & & & & & & \ck & \ck & \ck & & & & & & \\ \hline
	  G & & & & & & & & & & \ck & \ck & \ck & & & & \\ \hline
	  H & & & & & & & & & & & & \ck & \ck & \ck & & \\ \hline
	  Local & \multicolumn{4}{|c||}{Brasil}
    & \multicolumn{4}{|c||}{Outro País?}
    & \multicolumn{4}{|c||}{Brasil}
    & \multicolumn{4}{|c|}{Brasil} \\ \hline
\end{tabular}
\end{table}

Legenda: \ck~ a ser realizado; x  concluído.


\section{\sectionVI}
\label{sec:materiais-metodos}
Que materiais (placas de hardware, kits de desenvolvimento, itens de medição) são necessários para desenvolver sua pesquisa?

Quais os métodos que vc pretente utilizar para desenvolver sua pesquisa? Colocar este tipo de informação nesta seção.

\section{\sectionVII}
\label{sec:analise-resultados}
Como verificar/analisar se o que vc produziu está correto? \textit{Benchmark} disponível na literatura, estudos comparativos, reprodução de pesquisas? Como garantir a corretude dos resultados. Como verificar quão bom está o seu trabalho em relação ao que já existe atualmente?


\section{\sectionVIII}
\label{sec:pesquisa-exterior}
Eventual estágio no exterior dizendo onde, qual o professor supervisor e as devidas justificativas para posterior solicitação formal do estágio de pesquisa no exterior.

\section{\sectionIX}
\label{sec:general-notes}
If applicable, you should place here with relevantes notes needs to clarify any point of your research work.

% references
\bibliographystyle{ieeetr}
\bibliography{refs/references}

% todo list
\listoftodos

\end{document}